\documentclass[12pt]{report}

%----------------------------------------------------------------------
% Create a listing in the log of all files needed to process this document
 \listfiles

%----------------------------------------------------------------------
\usepackage{epsfig}
\usepackage{graphicx,color,psfrag}
\usepackage{amsmath}
\usepackage{subfigure}
\usepackage{longtable}
\usepackage{amssymb}
\usepackage{enumerate}
%\usepackage{array}
\usepackage{makeidx}
%\usepackage{showidx}
%\usepackage{multind}
%\usepackage{index}
\usepackage{pifont}
%\usepackage{verbatim}
\usepackage{moreverb}
%\usepackage{fancybox}
%\usepackage{boxedminipage}
\usepackage{calc}
\usepackage{rotating}
%\usepackage{bbold}

\newenvironment{Ventry}[1]%
   {\begin{list}{}{\renewcommand{\makelabel}[1]{\textsf{##1}\hfil}%
      \settowidth{\labelwidth}{\textsf{#1:}}%
      \setlength{\leftmargin}{\labelwidth+\labelsep}}}%
   {\end{list}}


 \setlength{\textheight}{8.75in}     % jan
 \setlength{\topmargin}{-0.5cm}     % jan
\setlength{\evensidemargin}{0mm}
\setlength{\oddsidemargin}{0mm}
\setlength{\textwidth}{7in}
\setlength{\textheight}{9.5in}
\setlength{\topmargin}{-20mm}
\newlength{\invisible}
\settowidth{\invisible}{1}
\raggedbottom

%................................................................

\begin{document}
\noindent

In class, we wrote on the blackboard a byte addressable memory where each element was 2 nibbles:
For example:
\newline

Main memory A
\tiny
\begin{center}
\begin{tabular}{ |c||c|c|c|c|c|c|c|c|c|c|c|c|c|c|c|c| }\hline
Address & Data & Data & Data & Data & Data & Data & Data & Data & Data & Data & Data & Data & Data & Data & Data & Data \\ \hline
Offset  &  0 & 1 & 2 & 3 & 4 & 5 & 6 & 7 & 8 & 9 & A & B & C & D & E & F \\ \hline
0  &  0x00 & 0x02 & 0x2B & 0x4F & 0x00 & 0x00 &0x00 &0x1C & 0x00 & 0x00 & 0x01 & 0x00 & 0x05 & 0x04 & 0x03 & 0x02 \\ \hline
0x10  & 0x10 & 0x10 & 0x11 & 0x12 & 0x00 & 0x00 &0x00 &0x00 & 0x00 & 0x00 & 0x01 & 0x00 & 0x3D & 0x00 & 0x1C & 0x2F \\ \hline
0x20  & 0x00 & 0xFF & 0x3E & 0x00 & 0x00 & 0x00 &0x00 &0x00 & 0x00 & 0x00 & 0x01 & 0x00 & 0x1F & 0xFF & 0x03 & 0x02 \\ \hline
\hline
\end{tabular}
\end{center}

\normalsize
\normalsize

What is the contents of address 0x1C in main memory A for a 32 bit machine
using Big Endian format?
\newline
\newline

What is the contents of address 0x1C in main memory A for a 16 bit machine
using Little Endian format?
\newline
\newline

What is the contents of the indirect address at 0x04 in main memory A for a Big Endian 32 bit machine ((0x4))? 
\newline
\newline

What is the contents of 4(0x10) in main memory A for a 16 bit Little Endian machine?  
\newline
\newline

What is the contents of the address 16(0xC) for  a 64 bit Little Endian machine?


\end{document}
